\errorcontextlines9999
\documentclass[a3paper, portrait, english, default]{uulm-cs-poster}

\usepackage{lipsum}

\title{flowR}
\subtitle{Layout Preserving Reconstruction}
\author{Pascal Deusch\and Wizard 2}
\institute{Institut für Software Engineering und Programiersprachen}
\university{Universität Ulm}
\logo{\splogo}
\date{\today}

\addbibresource{references.bib}
\nocite{*}

\begin{document}
\maketitle
%\section*{Why dogs are better cats}
%\lipsum[2]
%\vfil
\begin{multicols}{2}
\section*{FlowR}
   %Describtion of FlowR
	flowR is a static dataflow analyzer and program slicer for the R programming language.\\%reference flowR repo
	flowR allows its users to look at dataflow graphs for their R programs, generate slices for specified variables, and reconstruct those slices, amongst other functionalities.
	flowR is build in modules with its main modules being:
   \begin{itemize}
      \item a statistics module
      \item a benchmark module
	  \item a core module
      \item and a slicer module.
   \end{itemize}
	The statistics module is used to analyze R file and identify common patterns in them. The benchmarker module serves as a way to benchmark flowR's performance.
	The core module contains flowR's main definitions and is the home of flowR's read-eval-print loop (REPL) and flowR's server. The slicer module contains flowR's backward program slicer for R. This module consists of four submoduls with a custom R bridge, a way to normalize the AST representation of the dataflow graph, a dataflow analyzer for said dataflow graph, and the reconstruct module.
	My project is focused on the reconstruction module. There the work was mostly centered around the implementation of a layout preserving reconstruction.
\section*{Layout Preserving Reconstruction}
   \lipsum[4-6]
   \paragraph{Paragraph}\lipsum[7]
\end{multicols}
%\lipsum[2]
\section*{My Contributions}
\begin{multicols}{4}
   \lipsum[2]
   \printbibliography
\end{multicols}
\end{document}